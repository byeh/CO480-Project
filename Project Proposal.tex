\documentclass[11pt]{article}

\usepackage{amsmath,amsthm,amssymb}
\usepackage{hyperref}
\setlength\parindent{0pt}
\usepackage{setspace}
\usepackage{enumitem}
\setlist{noitemsep} %set line separation of items to zero
\setlength\parindent{0pt} %no automatic indentation for new paragraph
\addtolength{\textwidth}{4cm} \addtolength{\textheight}{3cm}
\addtolength{\topmargin}{-1.8cm}
\addtolength{\oddsidemargin}{-2cm}
\addtolength{\evensidemargin}{-2cm}
%\setlength{\parindent}{0cm}

\begin{document}

\begin{center}
{\Large CO 480 Project Proposal -- Spring 2015} 
\end{center}

\subsection*{Group Membership}
\begin{tabular}{|l|l|l|} \hline
\textbf{Name} & \textbf{ID} & \textbf{Emai}l \\ \hline
Michael Ryan Blair & 20410326 & mrblair@uwaterloo.ca \\ \hline
Xin Ye Liu & 20413393 & xy5liu@uwaterloo.ca\\ \hline
Brandon Yeh & 20453468 & byeh@uwaterloo.ca\\ \hline
Mohammed Tahir Zaman & 20392866 & mt2zaman@uwaterloo.ca\\ \hline
\end{tabular}

\subsection*{Project Summary}
\begin{tabular}{ll}
{\bf Person}  & John Forbes Nash Jr. \\
{\bf Place}   & United States of America, Princeton University\\
% Identify a place in time and geographically
{\bf Problem} & Equilibria in Strategic Games \\
{\bf Hook }   & This is the story of a man who achieved equilibrium\\  & between players who were not cooperative in games.
% A hook is a single sentence that grabs the attention of the reader/viewer. It might be the first sentence in the essay or in a verbal introduction.
\end{tabular}	

\subsection*{Project Outline}
% The more detail you can provide, the better.
% You may change this as necessary.

\paragraph{America through World War II, Cold War, and Civil Rights Eras}
\begin{enumerate}
	\item The end of World War II
	\item Liberalization of Trade \& Rebuilding of Europe
	\item Communism vs Capitalism
	\begin{enumerate}
		\item Creation of the Eastern Bloc
		\item Nuclear Arms Race --- Mutually Assured Destruction
		\item Korean War
		\item McCarthy \& the Red Scare
	\end{enumerate}
	\item Continued Development of Game Theory and its Applications
	\item Beginning of Civil Rights Movements
\end{enumerate}

\paragraph{John Forbes Nash, Jr.}
\begin{enumerate}
	\item Early Life
	\item Undergraduate Studies ---  From Chemical Engineering to Mathematics
	\item Princeton \& Thesis
	\item Interests other than Game Theory
	\item Family Life
	\item Struggles with Mental Illness
	\item Nobel Prize \& Other Recognitions
	\item Recent Exploits - Agency in Game Theory
\end{enumerate}

\newpage
\paragraph{Equilibria in Strategic Games}
\begin{enumerate}
	\item Introduction to Strategic Games
	\item Best Response Functions
	\item Pure Equilibria \& Cournot Oligopoly
	\item Mixed Equilibria
	\item Existence of Mixed (Nash) Equilibria
	\begin{enumerate}
		\item Sperner's Lemma
		\item Browser's Fixed Point Theory
		\item Nash's Existence Proof
	\end{enumerate}
	\item Practical Applications
	\item Lemke-Howson Method for Finding Equilibria
\end{enumerate}


\subsection*{Source Material}

% Preliminary identification of source material.
% Your source material will evolve with the project.
% Most of these sources should be candidates for the Annotated Bibliography.
% You should have at least five sources of which at least three are not web sources 

\begin{enumerate}
\item Binmore, K. (2011). Commentary: Nash's work in economics. \textit{Games and Economic Behaviour}, 71(1), 2-5.
\item Cook, M. R. (2009). \textit{Mathematicians: an outer view of the inner world.} Princeton, NJ: Princeton University Press.
\item Gaddis, John Lewis.  \textit{The Cold War: A New History}. New York: Penguin, 2005. Print.
\item Hart, S. (2011). \textit{Commentary: Nash equilibrium and dynamics. Games and Economic Behaviour,} 71(1), 6-8.
\item Mccain, K. W., \& Mccain, R. A. (2010). Influence \& incorporation: John Forbes Nash and the “Nash Equilibrium”. \textit{Proceedings of the American Society for Information Science and Technology}, 47(1), 1-2.
\item Meltzer, H. (1999). A Beautiful Mind: A Biography. \textit{The Journal of Clinical Psychiatry}, 60(4), 266.
\item Nasar, S. (2001). A Beautiful Mind: \textit{The Life of Mathematical Genius and Nobel Laureate John Nash}. New York: Simon \& Schuster.
\item Nash, John F. \textit{Non-cooperative Games}. Thesis. Princeton University, 1950.
\item Nosal, E., \& Rupert, P. (2002). A beautiful theory. \textit{Federal Reserve Bank of Cleveland}, 1-4.
\item Saint-Laurent, P. (n.d.). Beautiful minds: The competitive world of financial planning meets the mathematical.\textit{ Advisor's Edge}, 5(6), 45.
\item Truman, Harry. "Truman Library - Marshall Plan Online Research File."  \textit{Truman Library - Marshall Plan Online Research File}. The Harry S. Truman Library and Museum, 20 May 2015. Web. 20 May 2015.  $<$ \url{http://www.trumanlibrary.org/whistlestop/study_collections/marshall/large/index.php} $>$ .
\item Young, H. P. (2011). Commentary: John Nash and evolutionary game theory. \textit{Games and Economic Behaviour}, 71(1), 12-13.
\end{enumerate}

\end{document}

