\documentclass[11pt]{article}
\usepackage[hyphens]{url}
\usepackage{amsmath,amsthm,amssymb}
\usepackage{hyperref}

\addtolength{\textwidth}{4cm} \addtolength{\textheight}{3cm}
\addtolength{\topmargin}{-1.8cm}
\addtolength{\oddsidemargin}{-2cm}
\addtolength{\evensidemargin}{-2cm}
%\setlength{\parindent}{0cm}

\begin{document}

\begin{center}
{\Large CO 480 Editorial Review -- Spring 2015} 
\end{center}

\subsection*{Project Information} % Leave this in!
\begin{tabular}{ll}
{\bf Person}  & Isaac Newton \\
{\bf Place}   & (not obvious) \\
{\bf Problem} & Discovery of Calculus \\
\end{tabular}

\subsection*{Summary and Evaluation}

\begin{tabular}{ll}
{\bf Completeness - Person}  & 1 \\
{\bf Completeness - Place}   & 0 \\
{\bf Completeness - Problem} & 1 \\
{\bf Evaluation}             & Needs Lots of Work \\
\end{tabular}\\

Overall, this report needs significant work in both content and quality of writing. While the project describes in sufficient detail the life of Isaac Newton prior to his mental breakdown in 1693, it leaves major gaps in the history of the person. Additionally, there is no clear place section that identifies the historical backdrop to Newton's work. Though the authors attempted to describe the need for calculus beginning from Ancient Greece, this section is simply a restatement of the course lectures. Furthermore, each individual section of this history attempt is not in-depth enough to constitute a description of the place and background of Newton's work. With respect to the mathematics, this paper merely presents simple concepts (e.g. Binomial Theorem) already presented numerous times in core mathematics classes. These concepts are not treated in a new way -- they are simply copies of existing works. While there is some promise when the report deals with applications of calculus, no section developed thoroughly enough to present new mathematical knowledge to the reader.\\

In terms of quality of writing, the report leaves much to be desired. It has numerous grammatical errors (inconsistent capitalization, incorrect punctuation, etc.) as well as many typographical errors. Many sections do not add value to the report and are difficult to follow. A more detailed outline may help the authors reconstruct the report in a way that will flow better to present an easier to follow report. Although the report has a number of valuable sources, many passages are directly quoted or lifted from sources without a citation. Finally, the report has large portions of whitespace, and even then only uses 16 of the possible 20 pages, leaving a large amount of room to provide additional details and expand on the topics presented.

\newpage
\subsection*{Fact Checking}
% If all of the facts check out then write something like
% If there are notes place them in a list like this.
% Use + to count lines from the top of a page and -- to count lines from the bottom
Most facts are correct, and there are few major omissions. The following are omissions, that if included, would enhance the report.
\begin{itemize}
\item Omission, Page 1, Line +4. While the date of Newton's birth is accurate according to some records, the report does not discuss the change from the Julian to the Gregorian calendar which has caused some ambiguity as to the exact date of birth. Alternate date is January 4, 1643. See \url{http://www.mathpages.com/home/kmath121/kmath121.htm}.

\item Omission, Page 3, Line +10. While Newton's Principia was not printed until 1687 as stated, it was actually presented to the Royal Society and accepted for publication early in 1686 and a series of discussions and politicking delayed its publication. A discussion of these reasons would have enhanced the report. See \url{http://users.clas.ufl.edu/ufhatch/pages/13-NDFE/newton/05-newton-timeline-m.htm}.

\item Omission, Page 13, Section 3.1.4. This section is missing significant depth on the dispute between Leibniz and Newton regarding the invention of the calculus. Specifically, the authors present a case that lays the root of the conflict squarely at the feet of Leibniz. However, this was not the case. A number of other actors were involved with the dispute, and this passage does a disservice to the reader in not providing the full context of the dispute. See \url{http://pages.cs.wisc.edu/~sastry/hs323/calculus.pdf}.
\end{itemize}

\subsection*{Editing}

\begin{itemize}
\item Rewriting these parts to improve clarity and readability
	\begin{itemize}
		\item Page 2, Line +12, +13: Replace ``As a sizar, even though ..." with ``Though his family was quite wealthy, Newton had to work as a sizar, like a servant, for his classmates".
		\item Page 3, Line +13, +14, +15: The sentence ``During his collegiate years ..." is a run on sentence. Consider splitting after ``advanced mathematics of the day".
		\item Page 2, Line --5,--4,--3,--2: The sentence ``Although there were ..." is a run=on sentence. Consider splitting after ``advanced mathematics of the day".
		\item Page 5, Line +12, +13: The sentence ``Making philosophies as ..."  does not make sense on its own. Consider combining the sentence prior to it. ``In contrast to Plato's methodologial approach to biology by performing experiments and observations, Aristotle approached physics from a more philosophical manner".
		\item Page 5, Line +15: Consider rewriting ``these philosophies were taken as the truth by people" as ``people took these philosophies as the truth for".
		\item Page 5, Line +15: Change ``range from" to "span".
		\item Page 5, Line+16: Insert Oxford comma to change ``logic and biology" to `logic, and biology".
		\item Page 5, Line --18: The sentence "For example, the root ..." does not make sense.
		\item Page 5, Line --14: The sentence ``... motion on an object ... acting upon it" should read ``another object acting upon a given object causes motion on the other object"/
		\item Page 5, Line +1: The sentence ``Various views ..." is a run-on sentence. Consider splitting at ``Ptolmey".
		\item Page 6, Line +1,+2,+3: Rewrite the sentence "Various views..." as "For over a thousand years, various views existed about the arrangement of the Earth, sun, planets and stars".
		\item Page 6, Section 2.1.2: Change the title heading from ``Islamic Golden Age" to something such as ``The Awakening of Europe" or include additional information about Islamic scholars as the section does not currently focus on Islamic scholars.
		\item Page 6, Line --1: The ending of the sentence ``... being seemingly true`` is awkward.
		\item Page 8, Line +4, +5: The sentence ``Even in relatively low speed... launching off an already moving object is to be left behind" is wordy and confusing.
		\item Page 8, Line +21: It's not clear which book is being referred to.
		\item Page 9, Line +1, +2, +3: This sentence does not make sense.
		\item Page 11, Line +2, +3: ``The discovery of tangents were most important ... accelerations, for example." Try to reword this to be more direct.
		\item Page 13, +5, +6: It is not clear why being useful in understanding partial derivatives makes Leibniz notation the standard notation for calculus.
		\item Page 14, Line --2: It is not clear how having different notation resulted in the accreditation that they both invented calculus independently.
		\item Page 15, Line +1: This sentence is incomplete since it does not express a full thought.
		\item Page 15, Line --2, --3, --4:``Primarily dealing with ... which creates chemical compounds" is a sentence fragment.
		\item Page 16, Line +10: The paragraph jumps abruptly from ``securities" to ``bonds".
	\end{itemize}
	
	\item Remove extraneous words and reduce wordiness. This will also minimize the passive voice in the writing.
	\begin{itemize}
		\item Page 1, Line --2: Remove ``off".
		\item Page 1, Line --1: Remove ``any".
		\item Page 2, Line +4: ``had to go sell the produce of the farm to the market" is wordy.
		\item Page 2: Line +7: replace ``Headmaster, Mr. Stokes" with ``Headmaster Stokes".
		\item Page 2, Line +7: replace ``they were able to convince" with ``convinced".
		\item Page 2, Line +8: replace ``let Newton return" with ``let him return".
		\item Page 2, Line +15: replace ``would later write" with ``later wrote".
		\item Page 2, Line --16, --17: Rewrite as ``Studying Rene Descartes' `Geometry' independently led Newton down a very mathematically-concentrated path.".
		\item Page 2, Line --15: Replace ``due to the fact that" with ``since".
		\item Page 2, Line --6: Replace ``During the time of ..." with ``Newton's greatest discoveries were during The Plague".
		\item Page 2, Line --5: Replace ``Although there were ... best mathematicians" with ``Although questions remained unanswered even by the best mathematicians".
		\item Page 3, Line +1, +2: Rewrite as ``...Fermat overcame this issue but sacrificed beauty, elegance, and simplicity."
		\item Page 3, Line +3: Rewrite as ``In the spring of 1665, Newton discovered calculus."
		\item Page 3, Line +8: Remove ``for" and ``between".
		\item Page 3, Line --7: Replace ``was able to publish" with ``published".
		\item Page 3, Line --5: Remove ``that".
		\item Page 3, Line --4: Remove both instances of ``that".
		\item Page 4, Line +8: Remove ``we live in".
		\item Page 4, Line +9: Remove ``better" and ``had".
		\item Page 4, Line +11: Replace ``have been" with ``were".
		\item Page 4, Line +12: Remove ``to be".
		\item Page 4, Line --13: Replace ``which" with ``that".
		\item Page 4, Line --6: Replace ``on the basis of" with ``based on".
		\item Page 4, Line --2: Remove ``which could be used".
		\item Page 5, Line +1: Remove ``is something that".
		\item Page 5, Line +14: Remove ``is now".
		\item Page 5, Line +16: Remove ``up".
		\item Page 5, Line --13: Replace ``was not able to" with ``did not".
		\item Page 5, Line --11: Replace ``up to" with ``until".
		\item Page 5, Line --10: Remove ``had".
		\item Page 5, Line --6: Remove ``also".
		\item Page 5, Line --3: Replace ``it was believed that" with ``they believed".
		\item Page 6, Line +9: Remove ``time".
		\item Page 6, Line +11: Remove ``also".
		\item Page 6, Line --5: Remove ``then" and replace ``into" with ``to".
		\item Page 6, Line --5: Replace ``the Aristotelian tradition ... years" with ``the thousand-year old Aristotelian tradition".
		\item Page 7, Line +14: Remove ``was able to" and replace ``formulate" with ``formulated".
		\item Page 7, Line --6: Remove ``had".
		\item Page 7, Line --3: Remove ``great" since it's superfluous.
		\item Page 8, Line +1: replace ``proved to generate" with ``generated".
		\item Page 8, Line +3: Replace ``there was no coherent explanation as to why objects did not simply fly off the face of the Earth." with ``objects should simply fly off the face of the Earth."
		\item Page 8, Line +6: remove the word ``did", replace ``move" with ``moved".
		\item Page 8, Line +7: Replace ``was" with ``were".
		\item Page 8, Line +9: Replace ``bodies which" with ``bodies that".
		\item Page 8, Line +14: Replace ``challenge .. the response" with ``challenge to which Newton found the response".
		\item Page 8, Line +15: Remove ``be allowed to".
		\item Page 8, Line +19: Replace ``saw" with ``placed".
		\item Page 10, Line +7, +8: ``It is of note that Newton refers to rectangles as parallelograms." The sentence begins with too many words.
		\item Page 12, Line 1: This sentence is purely an opinion, and adds no value.
		\item Page 12,Line +11: ``at around the exact time" is redundant.
		\item Page 12, Line +13: Remove ``as well".
		\item Page 12, Line +17: Replace ``had studied" with ``studying".
		\item Page 12, Line --8: Replace ``Contrary" with ``Unlike".
		\item Page 12, Line --7: Remove ``himself".
		\item Page 12, Line --1: Replace ``which" with ``that".
		\item Page 13, Line +2, +3: Remove ``very" and ``very" (both instances).
		\item Page 13, Line +4: Remove ``also" and ``to" and ``to" (both instances).
		\item Page 13, Line +6: Remove ``very".
		\item Page 13, Line +7: Remove ``very".
		\item Page 13, Line --12: Remove ``or not".
		\item Page 13, Line --6: Remove ``form of the".
		\item Page 13, Line --5: Replace ``truth would be revealed two hundred years later that" with ``truth revealed two hundred years later showed".
		\item Page 14: Remove the final paragraph. It doesn't flow with this section and adds no value.
		\item Page 15, Line 2: remove ``had".
		\item Page 15, Line +8: Remove ``and foremost".
		\item Page 15, Line +13: Replace ``have become" with ``were".
		\item Page 15, Line +13: Replace ``, and are" with ``and".
		\item Page 15, Line --8. Replace ``the times of Ancient Greece, it would have been considered" with ``Ancient Greece, this was".
		\item Page 16, Line +7. This first sentence adds no value.
		\item Page 16, Line --6. Replace ``A notable concept" with ``The concept of is the".
		\item Page 16, Line --5. Remove ``, which".
	\end{itemize}
	
	\item Please be careful with punctuation usage
	\begin{itemize}
		\item Page 2, Line +18. Insert a comma after ``Newton".
		\item Page 2, Line --17. Insert a comma after ``religious man".
		\item Page 6, Line +2. Insert commas after ``Ptolemy" and ``Greek era".
		\item Page 9, Line --9. The comma before ``In this case" should be a period.
		\item Page 10, Line +2. Add a comma after ``today".
		\item Page 12, Line +6. Add a comma after ``Cambridge University".
		\item Page 12, Line +9, +10. ``These fields of study ... on the use of calculus to understand." Remove the semicolon. 
		\item Page 12, Line --3. replace ``mathematics, calculus" with ``mathematics. calculus.".
		\item Page 13, Line +5. replace ``Leibniz" with ``Leibniz's".
		\item Page 13, Line +16. replace ``differential calculus," with ``differential calculus.".
		\item Page 15, Line +1. remove the period.
		\item Page 15, Line --3. Insert a comma after ``bonds".
		\item Page 16, Line --9. Insert a comma after ``Thus".
		\item Page 16, Line --5. Remove the comma after ``understood".
	\end{itemize}
	
	\item Italicize the titles of books and other works
	\begin{itemize}
		\item Page 2, Line +15
		\item Page 2, Line +24
		\item Page 2, Line +25
		\item Page 2, Line --19
		\item Page 3, Line --7
		\item Page 4, Line --7, --10
		\item Page 8, Line --5, --6
		\item Page 13, Line --6		
	\end{itemize}
	
	\item Don't use first first person point of view
	\begin{itemize}
		\item Page 12, Line +9. Replace ``as we know today;" with ``as known today,".
		\item Page 12, Line --1. Replace ``what we now know" with ``now known".
		\item Page 15, Line +3
		\item Page 15, Line +2
	\end{itemize}
	
	\item Typographical Errors
	\begin{itemize}
		\item Page 1, Line --6: Replace ``reasons" with ``reason".
		\item Page 2: Line +11: Replace ``Bachelor" with ``Bachelor's".
		\item Page 2, Line --20: Replace ``astrology" with ``astronomy".
		\item Page 4, Line +3: Replace ``to" with ``do".
		\item Page 5, Line +7: Replace ``academics" with ``academic".
		\item Page 8, Line +20: Replace ``phenomenon" with ``phenomena" to fix the cardinality.
		\item Page 12, Line --1: Replace ``a genius" with ``an ingenious" since it should be an adjective not noun.
		\item Page 13, Line +14: Replace ``war on calculus" with ``calculus war". There was not a war against calculus, only a controversy on the discoverer of calculus.
		\item Page 15, Line +10: Replace ``Astrology" with ``astronomy".
		\item Page 16, Line --9: Replace ``behavior" with ``behaviour" to use Canadian spelling instead of American spelling.
	\end{itemize}
	
	\item Unnecessary capitalization
	\begin{itemize}
		\item Page 15, Line +6: ``Physics, Astrology and Chemistry"
		\item Page 15, Line --8: ``Heavens"
		\item Page 15, Line --9: ``Solar System"
		\item Page 15-16: Any reference to an academic subject, such as ``Biology", ``Zoology", ``Physics", ``Life Science".
		\item Page 16, Line +10: ``Series"
	\end{itemize}
	\item Use formal language instead of colloquial language
	\begin{itemize}
		\item Page 2, Line +4: ``Learn the ropes" is colloquial.
	\end{itemize}
	\item Additional suggested changes
	\begin{itemize}
		\item Page 2, Line +16: Replace ``his bachelors" with ``his bachelor's degree".
		\item Page 2, Line --13: Replace ``begins" with  ``started".
		\item Page 2, Line --14: The sentence ``These were questions ..." is missing punctuation or a clause. It does not make sense in its current form.
		\item Page 3, Line +1: Replace the question mark with a period.
		\item Page 3, Line +4: Replace ``fascination to" with ``fascination or".
		\item Page 3, Line --6: Write out the number 5 as ``five".
		\item Page 4, 2.1 Title: Replace ``1600's" with ``1600s".
		\item Page 4, Line +7, +11: Replace ``prior to" with ``before".
		\item Page 4, Line --6: Replace ``may" with ``might".
		\item Page 5, Line +2: Replace ``allegedly having ... written" with ``allegedly writing ....".
		\item Page 5, Line +17: Replace ``who" with ``that".
		\item Page 6, Line +12: Replace ``created basic" with ``created the basic".
		\item Page 6, Line --11: Write out ``twelfth" instead of putting $12^{\text{th}}$.
		\item Page 7, Line +5: Replace ``have" with ``had". The verbs need consistent tense.
		\item Page 8, Line +4: Replace ``in relatively low speeds" with ``at relatively low speeds."
		\item Page 8, Line +15: The past perfect tense should not be used here.
		\item Page 8, Line +16: Replace ``No longer was the heavens" with ``No longer were the heavens."
		\item Page 8, Line --2, --3, --4, --5: The citation should appear in the same sentence.
		\item Page 8, Line --2: ``but instead followed strict laws" does not follow from the sentence structure established earlier. 
		\item Page 9, Line 7: Replace ``integrated calculus into his book" with ``included calculus into his book." The pun is not appropriate. 
		\item Page 13, Line --17: replace ``title" with ``entitled". 
		\item Page 14, Line +3: replace ``integral of the derivative" with ``integral and the derivative".
		\item Page 14, Line +2: replace ``had changed ... understood" to ``has changed ... understand." 
		\item Page 15, Line +11: Replace ``Newton had established" with ``Newton established." There is no need to use the past perfect tense. 
		\item Page 16, Line +15: replace ``of change in interest rate" with ``of a change in interest rate." 
		\item Page 16, Line +16: replace ``based off" with ``derived."
		\item Page 16, Line --2: Replace ``which" with ``that".
		
		
	\end{itemize}
\end{itemize}

\subsection*{Mathematics}

The mathematics is not targeted at an appropriate audience. The mathematics presented is simply restatements of theorems and notation that is content from basic mathematics classes (STAT 230, MATH 137, ACTSC 231). 

\begin{itemize}
\item Omission: Pages 6, 7, Diagrams: Include some interpretation of the diagrams, otherwise they are confusing for the reader and do not add value to the report.
\item Typographical error: Page 9, Line +17, $3^{\text{rd}}$ term on the right hand side is missing $x^2$.
\item Typographical error: Page 9, Line +17, $n$ should be $k$.
\item Omission: Page 9: Binomial Theorem - No proof is given, and the theorems are simply stated. This theorem in particular is well known to the target audience ($3^{\text{rd}}$ year UW Math students) and as such, the lack of proof or additional content makes this section irrelevant. Finally, no context is given for the bound on $x: -1 < x < 1$ given in Newton's Binomial Theorem.
\item Omission: Page 10, Line --1: following $x^{n-1}$ insert ``at the point $x$".
\item Error: Page 14, Line +6: Indefinite integral is missing the constant of integration.
\item Omission: Page 14, Line +6: No discussion is given on Leibniz \& Newton's notation for integration (even if it is the same, this would be a relevant point).
\item Phrasing: Page 14, Line --5: The presentation here makes it difficult to clearly see what is Newton's notation, and not Leibniz's. 
\item Error: Page 16, Line +14: report states that a Taylor Series determines the interest rate sensitivity of a bond, however it would be less misleading and more accurate to state that a Taylor Series approximates this sensitivity.
\item Omission: Page 16, Line +16: $i$ is not defined. It should be defined as the nominal interest rate.
\end{itemize}

\subsection*{Plagiarism}

There is some evidence of plagiarism.

\begin{itemize}
\item Uncited Extract: Page 1. extract from list of 48 sins is directly quoted, but not cited. Source: \url{http://www.huffingtonpost.com/2012/01/06/isaac-newton-list_n_1190714.html}.
\item Uncited Extract: Page 4 Line +2,+3, +4, +5. These are direct quotes which are not cited from \textit{Magnificent Principia: Exploring Isaac Newton's Masterpiece} by Colin Pask. Excerpt here: \url{https://books.google.ca/books?id=lRhnAAAAQBAJ&pg=PT29&lpg=PT29&dq=The+great+questions+driving+Newton\%E2\%80\%99s+Principia+are:\&source=bl&ots=DPIaUGTDaQ&sig=xLhunYvlgN3tKALUQBFlYjORxBo&hl=en&sa=X&redir_esc=y#v=onepage&q=The\%20great\%20questions\%20driving\%20Newton\%E2\%80\%99s\%20Principia\%20are\%3A&f=false}.
\item Uncited Extract: Page 5 Line +3. ``let no one ignorant of geometry enter here" is a direct quote that is uncited. Source: \url{http://plato-dialogues.org/faq/faq009.htm}.
\item Uncited Extract: Page 5, Line 20. ``a prerequisite for knowing anything is understanding why it is". Source: \textit{Physics} by Aristotle. \url{http://philosophyideas.com/search/idea_detail.asp?find=idea&visit=2&ThemeNumber=237&area=Society&area_no=25&ID=8331&return=yes&theme_alpha=yes&gistsfor=idea&source=theme}.
\item Uncited Diagram: Page 6, diagram. Diagram is copied from \url{https://mhs-integrated-curriculum.wikispaces.com/Beginning+-+ancient+civilizations} but is uncited.
\item Uncited Extract: Page 6, Line +18, +19, +20, +21. ``matters where reason and faith collided, faith must win" is a direct quote followed by an essential paraphrase of an uncited quotation. Source: Magnificent Principia: Exploring Isaac Newton's Masterpiece.
\item Uncited Extract: Page 7, Line +15, +16, +17, +18, +19, and diagram. Copied directly from \url{https://en.wikipedia.org/wiki/Kepler\%27s_laws_of_planetary_motion} with no citation.
\item Uncited Extract: Page 8, Line --6. ``discovery of causal relationships by systematic experimentation" is copied from \url{http://www.goodreads.com/quotes/1128357-development-of-western-science-is-based-on-two-great-achievements} but is uncited.
\item Uncited Extract: Page 8, Line --13. Galileo's quote needs to be in quotations and cited. 
\item Uncited Extract: Page 10, Lemma II. No source is given for this lemma and diagram.
\end{itemize}

\subsection*{References}

The references are correct and useful.

\begin{itemize}
\item Omission: Reference[15] is missing the URL: \url{http://www.famousscientists.org/gottfried-leibniz/}.
\end{itemize}


\end{document}

