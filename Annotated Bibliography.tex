\documentclass[11pt]{article}

\usepackage{amsmath,amsthm,amssymb}
\usepackage{hyperref}
\setlength\parindent{0pt}
\usepackage{setspace}
\usepackage{enumitem}
\setlist{noitemsep} %set line separation of items to zero
\setlength\parindent{0pt} %no automatic indentation for new paragraph
\addtolength{\textwidth}{4cm} \addtolength{\textheight}{3cm}
\addtolength{\topmargin}{-1.8cm}
\addtolength{\oddsidemargin}{-2cm}
\addtolength{\evensidemargin}{-2cm}
%\setlength{\parindent}{0cm}

\begin{document}

\begin{center}
{\Large CO 480 Annotated Bibliography -- Spring 2015} 
\end{center}

\subsection*{Group Membership}
\begin{tabular}{|l|l|l|} \hline
\textbf{Name} & \textbf{ID} & \textbf{Emai}l \\ \hline
Michael Ryan Blair & 20410326 & mrblair@uwaterloo.ca \\ \hline
Xin Ye Liu & 20413393 & xy5liu@uwaterloo.ca\\ \hline
Brandon Yeh & 20453468 & byeh@uwaterloo.ca\\ \hline
Mohammed Tahir Zaman & 20392866 & mt2zaman@uwaterloo.ca\\ \hline
\end{tabular}

\subsection*{Annotated Bibliography}

\begin{enumerate}

\item Binmore, K. (2011). Commentary: Nash's work in economics. \textit{Games and Economic Behaviour}, 71(1), 2-5.\\

This is a short commentary on Nash’s work surrounding non-cooperative games, along with the various events in his life that affected his work. Published in the journal Games and Economic Behaviour, Ken Binmore discusses Nash’s axioms for rational outcomes of a bargaining situation. Binmore puts Nash’s contributions into context by comparing his mathematical ideas, to the dependency on “negotiation skills” commonly believed before Nash’s time. There are also comparisons to Von Neumann’s minimax theorem of 1928. The article then goes on to discuss the importance of his work, as well as the accolades he accumulated throughout his career. While this article does not give many specifics critiquing Nash’s work, it does give the reader a good idea of the game theory field before Nash’s contributions, and the effects that his work had in various applications.\\

\item Mccain, K. W., \& Mccain, R. A. (2010). Influence \& incorporation: John Forbes Nash and the “Nash Equilibrium”. \textit{Proceedings of the American Society for Information Science and Technology}, 47(1), 1-2.\\

This is an article written or the American Society for Informaiton Science and Technology by two professors at Drexel University. In this article, they describe some of Nash’s work, but more so his impact on the field of game theory as a whole. They specifically look at the different citations that Nash as received over time, as well as his “obliteration” rate, a measure of how often Nash Equilibrium was mentioned in other works. The study found that as more time passed, Nash’s works were cited exponentially more. It was on a steady growth trend well into the 21st century. This paper really demonstrates the adoption and spread of Game Theory concepts over time.\\ 

\item Saint-Laurent, P. (n.d.). Beautiful minds: The competitive world of financial planning meets the mathematical.\textit{ Advisor's Edge}, 5(6), 45.\\

This is an article written by Pierre Saint-Laurent, for the periodical Advisor’s Edge. In it, he describes the possible financial decisions applications of Nash’s work. He especially dives into the example of radio frequency auctions in the United States, and how the US government can set a fair price for radio frequencies by cell phone networks, while making sure that the right frequencies end up in the right hands. The result was that the government brought in many auction experts, and armed with Nash’s theories, they devised an auction, which eventually raised over \$10 Billion for the government, and allocated the right frequencies to the right companies. While this article spends a lot of time discussing the different possible applications of Nash’s work, it is aimed at an advisory audience, and thus there is no quantitative content.\\ 

\item Hart, S. (2011). \textit{Commentary: Nash equilibrium and dynamics. Games and Economic Behaviour,} 71(1), 6-8.\\

This is a short commentary from the journal Games and Economic Behaviour. The commentary discusses Nash equilibria with a focus on the applications and extensions of Nash equilibria to other areas of Game Theory. As a source for details on Nash’s thesis and insight into his life, this source is of limited value. However, it is a good starting point for further investigation into the impact the Nash’s work had on the field of Game Theory. Additionally, the source gives a brief overview of Nash’s own work on bargaining games and applying his theories on non-cooperative games to cooperative games, illustrating Nash’s ability to see the impact of his work beyond what he wrote in his thesis.\\

\item Nash, John F. \textit{Non-cooperative Games}. Thesis. Princeton University, 1950.\\

The origin of this source is Nash’s 1951 thesis submitted to Princeton University. The thesis was presented by Nash to receive his PhD. This thesis defines non-cooperative games, equilibrium points, and the solvability of games. Nash gives his now famous proof of the existence of equilibria in non-cooperative games. Interestingly, as Nash himself notes, this is not his first proof of this theorem. He originally proved his theorem in 1950 using a slightly different method. This paper was invaluable in establishing the field of Game Theory as we know it today. The paper is extremely short by modern thesis standards, but it full with important results. Not only does Nash prove the existence of equilibria, but he also hints at the applications of this theory. This source gives the mathematical historian a first-hand account of Nash’s thoughts on Game Theory and the methods he used to achieve his results. That said, there are limitations in using this thesis as a source. Primarily, the thesis skips a number of steps that a newcomer to the field would need to fully grasp Nash’s thoughts. Additionally, the thesis gives no indication of the mathematical discovery Nash went through to develop this theory. Ultimately, this source is the backbone of the mathematics section of this paper, giving Nash’s own proof of his landmark theory.\\

\item Truman, Harry. "Truman Library - Marshall Plan Online Research File."  \textit{Truman Library - Marshall Plan Online Research File}. The Harry S. Truman Library and Museum, 20 May 2015. Web. 20 May 2015.  $<$ \url{http://www.trumanlibrary.org/whistlestop/study_collections/marshall/large/index.php}$>$ .\\

This is the Truman Presidential Library’s collection of documents relating to the Marshall Plan. This plan shaped the economic aid that helped rebuild Europe following World War II. The collection consists of over 50 documents that illustrate the thoughts and actions of key decision makers at the time. This gives the reader important historical context for the Marshall Plan and walks us through the decision making at the time. The collection presents the speeches, correspondence, and written reports of key US officials such as Presidents Hoover and Truman, and Secretary of State George Marshall. The documents cover the Marshall Plan from conception through approval to implementation and evaluation. When John Nash was developing his theories on non-cooperative games, the primary macro environmental event was the end of WWII and the subsequent economic boom in the US and recovery in Europe. This source allows us to better understand the historical setting in which Nash worked, and the potential influences on his work.\\

\item Gaddis, John Lewis.  \textit{The Cold War: A New History}. New York: Penguin, 2005. Print.\\
 
Gaddis tries to answer the questions about what was fearful about the USSR, the uniqueness of the Cold War, and the unusual methods in this war. This book was aimed at audiences who do not have living memories of the Cold War, especially since hindsigh
t shows that the USSR was not a formidable opponent in many respects. Gaddis shows that this perspective was not known during the actual Cold War. Unlike normal history books, this book is a th
ematic exploration of the cold war. Some of these themes include the fear of the unknown, ideological differences, unique battlefronts (like proxy wars,the moon landing), dormant weapons (mutually assured destruction, the stalemate, change in country 
autonomy, and might vs. right. Gaddis also devotes a section of the book for the various Western and USSR actors that played a significant role in this war.\\

\item Nasar, S. (2001). A Beautiful Mind: \textit{The Life of Mathematical Genius and Nobel Laureate John Nash}. New York: Simon \& Schuster.\\

Nasar explores John Nash’s life by splitting it into five chronological sections: his school life and early genius, then his early adult life where he experimented, then his stagnant academic career, then his deteriorating mental illnesses, followed by Nobel P
rize.  Nasar was highly candid: she went in depth into many areas, including Nash’s bout of sexual curiosity and subsequent firing from the RAND Corporation and his unsuccessful struggle for a Fields Medal or Bôcher Prize. It details many of the conflicts he 
faced, with both himself and his personal and professional
 friends. For instance, Nash tried renouncing his citizenship and was later deported from Switzerland and France, and how this behaviour raised concerns with the nominating committee while they were selecting the 1994 Nobel Prize winner. Nasar’s book is one of the most detailed narratives of John Nash’s life.\\ 

\item Young, H. P. (2011). Commentary: John Nash and evolutionary game theory. \textit{Games and Economic Behaviour}, 71(1), 12-13.

This is a short commentary from the Journal of Games and Economic Behaviour. The commentary discusses Nash's two major papers on bargaining and the existence of equilibrium and how his two 1950 papers laid "the foundations for the subsequent development of both cooperative and non-cooperative game theory". This article also discusses the future work that will be done on top of Nash's work by other people. More specifically, the author will show a method of "of developing Nash's mass-action approach and show how it can be brought to bear on one of his other great contributions, namely, the Nash bargaining solution".
\end{enumerate}
\end{document}